% !TeX root = ../libro.tex
% !TeX encoding = utf8
%
%*******************************************************
% Introducción
%*******************************************************

% \manualmark
% \markboth{\textsc{Introducción}}{\textsc{Introducción}} 

\chapter{Introducción}

La investigación de las aparentes anomalías en la abundancia superficial de litio (Li) observadas en el Sol y en cúmulos globulares estelares jóvenes dentro de contextos astrofísicos contemporáneos es muy prometedora para avanzar en nuestra comprensión de los mecanismos que influyen en el agotamiento del Li a lo largo de la evolución estelar. Este estudio profundiza en la intrincada interacción entre la mezcla rotacional y los efectos hidrostáticos rotacionales en estrellas de pre-secuencia principal (PMS) y de secuencia principal (MS) empleando simulaciones de modelos en rotación.\par 

La evolución del Li para modelos solares se examina bajo la influencia combinada de la Teoría de la Longitud de Mezcla (MLT) y el frenado magnético (MB), donde la intensidad del campo magnético ($B$) y la parametrización MLT ($\amlt$) juegan un papel clave.


-- ESTO HABRÁ QUE COLOCARLO EN SU SITIO CUANDO HABLEMOS DE LOS MODELOS CON B Y ALFA VARIABLE --


La evolución del Li para modelos solares se examina bajo la influencia combinada de la Teoría de la Longitud de Mezcla (MLT) y el frenado magnético (MB), donde la intensidad del campo magnético ($B$) y la parametrización MLT ($\amlt$) evolucionan dinámicamente con parámetros solares clave, como son la velocidad angular ($\Omega$), la temperatura efectiva ($\teff$) o la densidad ($\rho$). Se propone un enfoque novedoso, que evita valores fijos para estos parámetros y produce resultados coherentes con los datos observacionales.\par

Se generan modelos solares precisos, que reflejan la naturaleza dinámica de $B$ y $\amlt$ a lo largo de la evolución estelar, y se contrastan con datos observacionales de cúmulos abiertos jóvenes obtenidos a través del Gaia-ESO Survey (GES). La inclusión de valores variables de $B$ y $\amlt$ reproduce congruentemente resultados en línea con trabajos anteriores en los que estas aproximaciones se han abordado por separado. Damos un paso más al incorporarlos conjuntamente en nuestros modelos y estudiamos el efecto combinado que producen sobre la historia rotacional y las abundancias superficiales de Li en modelos solares, obteniendo resultados que siguen en línea con esos trabajos, y compatibles con los datos observacionales.\par

Los resultados sugieren un límite inferior robusto de 1.133 dex para las abundancias superficiales de Li en estrellas similares al Sol, alineándose con las observaciones solares y arrojando luz sobre el intrincado equilibrio de los procesos físicos que gobiernan el contenido de Li en las atmósferas estelares. Asimismo, obtenemos valores teóricos de $\amlt$ acordes con el intervalo [1.76, 1.78] obtenido en trabajos anteriores. A la vista de estos prometedores resultados, nuestros modelos ofrecen una alternativa consistente y completa a aquellos con valores fijos, y con un tratamiento aislado de $B$ y $\amlt$. Hemos conseguido obtener simultáneamente resultados, que aunque no coinciden exactamente con las medidas reales del Sol, son sin embargo compatibles con su abundancia superficial de Li, velocidad angular y valores $\amlt$ predichos.\par

\endinput
