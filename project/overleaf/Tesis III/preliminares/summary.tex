% !TeX root = ../libro.tex
% !TeX encoding = utf8
%
%*******************************************************
% Summary
%*******************************************************

\selectlanguage{english}
\chapter{Summary}

Investigating the apparent anomalies in lithium (Li) surface abundance observed in the Sun and young stellar globular clusters within contemporary astrophysical contexts holds significant promise for advancing our comprehension of the mechanisms influencing Li depletion throughout stellar evolution. This study delves into the intricate interplay between rotational mixing and rotational hydrostatic effects in pre-main sequence (PMS) and main sequence (MS) stars by employing simulated grids of rotating models. The Li evolution of solar models is scrutinized under the combined influence of Mixing Length Theory (MLT) and magnetic braking (MB), where the magnetic field strength ($B$) and MLT parameterization ($\amlt$) dynamically evolve with key solar parameters. A novel approach, avoiding fixed values for these parameters, is proposed, yielding results consistent with observational data.\par

Accurate solar models, reflective of the dynamic nature of $B$ and $\amlt$ throughout stellar evolution, are generated and tested against observational data from young open clusters obtained through the Gaia-ESO Survey (GES). The inclusion of variable $B$ and $\amlt$ values congruently reproduces results in line with previous work in which these approaches have been addressed separately. We go a step further by incorporating them jointly in our models and study the combined effect they produce on the rotational history and surface Li abundances in solar models, obtaining results that are still in line with those works, and compatible with observational data.\par

The findings suggest a robust lower limit of 1.133 dex for surface Li abundances in Sun-like stars, aligning with solar observations and shedding light on the intricate balance of physical processes governing Li content in stellar atmospheres. Likewise, we obtain theoretical values of $\amlt$ in accordance with the [1.76, 1.78] interval  obtained in previous works. In view of these promising results, our models offer a consistent and comprehensive alternative to those with fixed values, and with an isolated treatment of $B$ and $\amlt$. We have managed to simultaneously obtain results, which although are not exactly matching the Sun's actual measurements, are however compatible with its surface Li abundance, angular velocity and predicted $\amlt$ values.\par


% Al finalizar el resumen en inglés, volvemos a seleccionar el idioma español para el documento
\selectlanguage{spanish} 
\endinput
