% !TeX root = ../libro.tex
% !TeX encoding = utf8

\chapter{¿Por qué el Litio?}\label{ch:primer-capitulo}

\section{Introducción}
El estudio de las abundancias de Litio en las estrellas, particularmente aquellas similares al Sol, es crucial por varias razones. En primer lugar, el Litio es uno de los pocos elementos producidos en la nucleosíntesis del Big Bang. Su abundancia proporciona una prueba crítica para la teoría del Big Bang y nos permite sondear las condiciones del universo primitivo. Además, el Litio es sensible a las temperaturas estelares, y su tasa de destrucción aumenta rápidamente a temperaturas superiores a 2.5 millones de Kelvin, que son típicas para los interiores estelares.\par

En segundo lugar, la abundancia de Litio puede arrojar luz sobre la estructura interna y los procesos de mezcla dentro de las estrellas. En los modelos estelares estándar, se espera que el Litio se agote en el sobre convectivo exterior de una estrella debido a su transporte a capas más profundas y calientes donde se destruye. Sin embargo, las observaciones a menudo muestran más agotamiento de Litio del que predicen estos modelos, lo que sugiere procesos de mezcla adicionales. Estudiar las abundancias de Litio puede ayudar a refinar nuestra comprensión de estos procesos y mejorar los modelos estelares.\par

El estudio de las abundancias de Litio es particularmente relevante al examinar cúmulos abiertos. Los cúmulos abiertos son grupos de estrellas que se han formado a partir de la misma nube molecular gigante, lo que significa que comparten una edad y composición química inicial comunes. Esto los convierte en excelentes laboratorios para estudiar la evolución estelar y la nucleosíntesis. El contenido de Litio en estas estrellas puede proporcionar valiosos conocimientos sobre los procesos de mezcla interna y la edad del cúmulo. A medida que las estrellas de un cúmulo envejecen, su abundancia de Litio superficial disminuye debido a la mezcla y la quema. Al comparar las abundancias de Litio observadas en las estrellas de un cúmulo abierto con modelos teóricos, podemos estimar la edad del cúmulo y obtener conocimientos sobre la eficiencia de los procesos de mezcla.\par

Además, dado que todas las estrellas en un cúmulo inicialmente tienen la misma abundancia de Litio, cualquier diferencia observada debe ser debido a procesos que ocurren dentro de las estrellas. Esto nos permite investigar cómo factores como la masa, la temperatura, la rotación y la presencia de campos magnéticos afectan a la destrucción de Litio, contribuyendo a nuestra comprensión de los interiores estelares. Por lo tanto, el estudio de las abundancias de Litio en cúmulos abiertos juega un papel crucial en el avance de nuestro conocimiento de las estrellas y su evolución.\par

aaaa \gls{lcm} \acrshort{gcd} \Gls{latex}

\endinput
%--------------------------------------------------------------------
% FIN DEL CAPÍTULO. 
%--------------------------------------------------------------------

