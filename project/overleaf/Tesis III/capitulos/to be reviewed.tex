aaaa \gls{lcm} \acrshort{gcd} \Gls{latex}

---ESTO PARECE ESTAR FUERA DE SITIO --

Los denominados modelos estándar, modelos que incluyen la convección sólo como proceso de mezcla y no consideran otras opciones de transporte como la difusión o la pérdida de momento angular (Angular Moment Loss, AML), han sido los principales implicados en la elaboración de estas predicciones \cite{Sestito2005}. El Li se destruye a una temperatura $\tli \approx 2.5 x 10^6\; K$ cuando un átomo de Li colisiona con un protón produciendo dos átomos de He, algo que tiene lugar durante las reacciones protón-protón tipo II (P-P II), y por tanto se destruye directamente en las envolturas estelares cuando la temperatura en la base de la zona de convección (Base Convective Zone, BCZ) alcanza ese valor. El Sol en particular, y las estrellas de tipo solar en general, se caracterizan por tener una CZ que cubre gran parte del radio estelar durante la PMS lo que hace que su límite inferior supere $\tli$ \cite{Iben1965}. Este agotamiento se detiene durante la aproximación al inicio de la MS (Zero-Age Main Sequence, ZAMS) cuando la zona de convección retrocede y la temperatura en la BCZ es más fría que $\tli$. En los modelos estelares estándar sólo la masa y la composición química inicial determinan a qué distancia de la superficie estelar se alcanza la temperatura $\tli$, por lo que se espera que estrellas dentro de cúmulos con masa similar alcancen la ZAMS con iguales abundancias superficiales de Li. Además, también deberían mostrar una evolución del Li muy similar hasta su aproximación a la edad terminal de la MS (Terminal-Age Main Sequence, TAMS). Durante este mismo periodo los mecanismos de convección desencadenan un proceso de mezcla que homogeneiza la composición química de la envoltura convectiva, desde su límite inferior hasta la superficie estelar. Sin embargo, se han observado diferentes abundancias de Li en distintas poblaciones estelares \cite[][y referencias en las mismas]{Somers2014}.\par

-- HASTA AQUÍ --

